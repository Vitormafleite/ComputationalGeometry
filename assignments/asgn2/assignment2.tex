\documentclass{article}
\usepackage[utf8]{inputenc}
\usepackage[brazil]{babel}
\usepackage[T1]{fontenc}

\title{Tarefa 2}
\author{Vitor Manoel}
\date{\today}

\setlength{\parindent}{0pt}

\begin{document}

\maketitle

\section*{Questão 1}

a) Podemos entender $\theta$(a,b) como o arco entre os vetores a e b, para expressar isso usando $\theta$(a) e $\theta$(b), 
entendemos que essas expressões retornam os arcos com o vetor padrão do plano 2D (1,0), ou seja, a simples diferença $\theta$(a)-$\theta$(b)
parece resolver o problema, entretanto, alguns casos onde os vetores estão em certos quadrantes retornam resultados negativos, o que não desejamos,
para corrigir isso, limitamos a expressão para o intervalo [0,8), acrescentando resto de 8 no final da expressão, por fim temos:

\[
\theta(a, b) = (\theta(b) - \theta(a)) \bmod 8
\]

b)



\end{document}
